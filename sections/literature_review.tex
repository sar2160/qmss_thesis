
\chapter{Literature Review}
\label{literature_review}

\section{Spatiotemporal Modelling Applications}

Count-based spatial and spatiotemporal models are widely used in ecological fields e.g., for wildlife population and habitat studies \cite{cobi_2008}, as well as in the public health field for disease mapping \cite{schrodle_2011}. Intrinsic  Conditional AutoRegressive models (ICAR) are a spatial-only model with widespread application in public health context and a produce a measure of relative risk similar to the one used for this project \cite{wakefield2006disease}.   Spatial and spatiotemporal models used in these fields are typically at a low to medium level of spatial granularity, such as the regional model of the northwestern United States used in \cite{cobi_2008}. For studies set in urban areas some research has been done at the level of census tracts in an individual county, like Chun et., al's study on disparities in environmental hazards in Maricopa County \cite{chun_2012}. \par

Outside of the natural and environmental sciences, some forms of spatial and spatiotemporal modelling have found use in predictive policing research and industry use, given that "space–time interactions are deeply embedded both empirically and theoretically into many areas of criminology" \cite{li_2014}. The nature of policing also dictates most or all predictive modelling is done in urban areas and at a much finer scale of spatial granularity. Recently developed likelihood-based estimators for spatiotemporal counts have been fit on crime data from 138 census tracts in Pittsburgh, Pennyslyvania \cite{liesenfeld_2017}; Flaxman et. al., developed a forecasting algorithm for crime in Portland, Oregon at the granularity of 66,000 cells measuring a quarter of a square mile each \cite{flaxman_2018}.
 \par

Other than in law enforcement, spatiotemporal models have also found some applications in urban social science through traffic modelling. Cheng et. al., used a non-Bayesian approach to model and forecast traffic dynamics in London \cite{cheng_2012}. In this study, the authors pose and attempt to grapple with one of the fundamental and ongoing challenges of using data in the context of a Smart City; the volumes of data available are often so massive thay they either overwhelm existing modelling methods and/or offer an almost endless number of ways to further segment and structure the data for use. Cheng et. al, use a variety of non-Bayesian methods including likelihood-based statistical models and gradient-based neural networks, but find these methods incapable of capturing the full autocorrelation structure of a traffic networks. The tradeoff between the spatiotemporal complexity of a model and the computational feasibility of fitting the model in medium to large datasets is evident here. \par
