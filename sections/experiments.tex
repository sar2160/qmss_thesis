
\section{Experiments}
\label{experiments}

Both long- and short-term forecasting experiments were considered, as well as both neighborhood level spatial granularity and a larger city level spatial scheme.

\todo{Table of SML term and city/nta/grid characteristics}


\subsection{GPflow}

GPflow is a Python package developed for fitting Gaussian Process models that takes advantage of the autodifferentiation abilities of Google's TensorFlow to generate Maximum-A-Posteriori (MAP) estimates of the posterior distribution \cite{GPflow2017} \cite{tensorflow2015-whitepaper}. All models are converted into tensors and passed to Tensorflow's optimization algorithms that can run on Graphics Processing Units (GPUs) for fast and efficient computation. This cuts the time needed for each iteration significantly compared to running on a CPU and also doesn't restrict the user to using conjugate distributions when specifying priors and likelihoods - a prerequisite for the LGCP model. GPflow is also capable of full Bayesian inference using Hamiltonian Monte Carlo (HMC).

\subsubsection{Custom Modifications}

While GPflow offers most of the settings required 'out-of-the-box', a few additional features had to be added independently. The base package does not offer the option for a Student-T prior, which was relatively easy to add by writing a new custom Student-T class to GPflow. The package also by default does not easily allow for decomposition of the different kernels' contribution to the parameter estimates. Finally, the GPflow implementation of Matern kernels often exhibit unstable behavior and, even after re-centering data. A small jitter was added to the Matern32 kernel in in order to reduce the chance of the Cholesky Decomposition failing. All modifications can be found in the appendix \todo(cite to appendix code).

\subsection{Kernel Choice}

Besides making a choice for or against stationarity, kernels also embed other assumptions about the model's
