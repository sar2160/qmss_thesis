
\chapter{Introduction}
\label{introduction}

Urban areas produce vast amounts of data each day, much of it bound to specific location and time. Every bus signaling its location to an app, every complaint made about noisy neighbors, every police report about a traffic collision is a measurement. While it easier than ever to put data on a map, making sense of it as an administrative or policy professional can be an exercise in frustration. Answers to simple questions about why things are happening and where they may happen next typically require statistical knowledge and access to even more explanatory data. \par

There is potential for explanatory or predictive modelling using solid statistical foundations and only the data at hand, and some progress in this direction has been made when considering timeseries. Facebook's Prophet is a successful forecasting tool with a Bayesian underpinning that nevertheless is accessible to analysts and researchers who bring their own skillset and expert knowledge \cite{prophet}. To use Prophet all one has to do is bring a timeseries and maybe some prior guesses about trends or important events. The additive model underpinning Prophet is capable of neatly decomposing the time series into multiple interpretable trends and also offers forecasting under uncertainty.

Using space as a predictive dimension in the same way Prophet uses time has the potential to be immensely helpful for urban research and administration. There is obvious theoretical and practical appeal to using spatiotemporal methods in the context of the 'Smart City', whose premise is found in rethinking urban areas as an intricate system monitored and managed at a previously impossible level of detail through the use of data \cite{kitchin_2014}. While Smart City proponents have pointed to the potential of technological data generating mechanisms such as sensor networks, there is already a vast amount of generated data available through administrative channels such as 311, public safety reporting, or existing networked sensor systems like traffic monitoring. While there have been limited surveys into the use of spatiotemporal models for urban applications, even these were restricted to forecasting at a low level of spatial granularity such as load demand for energy grids \cite{tascikaraoglu_2017}. Relatively little work has assessed the viability of making use of hyperlocal data for forecasting outcomes in an urban environment, much less productionizing a model to better deliver social services exactly where they are needed and at the right time.
